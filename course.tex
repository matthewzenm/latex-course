\documentclass{beamer}

\usepackage[fontset=none]{ctex}
\usepackage{verbatim}
\usepackage{geometry}
\usepackage[shortlabels]{enumitem}
\usepackage{fancyvrb}

\setCJKmainfont[BoldFont={FZCSJ.otf},
    ItalicFont={FZKTJ.otf},
    BoldItalicFont={FZCKJ.otf}]{FZSSJ.otf}
\setCJKsansfont[BoldFont={FZCSJ.otf},
    ItalicFont={FZKTJ.otf},
    BoldItalicFont={FZCKJ.otf}]{FZHTJ.otf}
\setCJKmonofont{FZFSJ.otf}

\usetheme{Warsaw}
\usefonttheme[onlymath]{serif}

\title{高速\LaTeX{}入门课程}
\author{曾梦辰}
\institute{北京师范大学OM学社}
\date{最后编译:\ \today}

\begin{document}
\maketitle

\begin{frame}
    \frametitle{关于\LaTeX{}}
    LaTeX (读作Lah--tek或Lay--tek, 不要读成Lay--teks), 是一种基于TeX的排版系统, 由美国计算机科学家Leslie Lamport在20世纪80年代初期开发.
    利用这种格式系统的处理, 即使用户没有排版和程序设计的知识也可以充分发挥由TeX所提供的强大功能, 不必一一亲自去设计或校对, 能在几天, 甚至几小时内生成很多具有书籍质量的印刷品生成复杂表格和数学公式, 这一点表现得尤为突出.
    因此它非常适用于生成高印刷质量的科技和数学, 物理文档.
    这个系统同样适用于生成从简单的信件到完整书籍的所有其他种类的文档.

    LaTeX使用TeX作为它的格式化引擎, 当前的版本是\LaTeX{2$\varepsilon$}.\pause
    \vspace{1cm}
    以上是抄的维基百科.
\end{frame}

\begin{frame}
    \frametitle{听完本次课程后, 你将会\dots}
    \begin{enumerate}[(1)]
        \item 学会如何使用Overleaf或者\TeX{}Page进行\LaTeX{}写作.
        \item 学会编写属于自己的\emph{Hello, World!}文档.
        \item \emph{知道}怎么输入数学公式.
        \item 学会如何使用\Verb|texdoc|命令或者在\Verb|CTAN|上查找宏包文档.
        \item 从我的群里薅走一些实用书籍.
    \end{enumerate}\pause
    我觉得这些就够了.
    学习\LaTeX{}不是听一次课就能学会的, 需要自己下来勤加练习.
    有了上面这$5$点, 我认为已经有了一个良好的开端.
\end{frame}

\begin{frame}
    \frametitle{\TeX{}Live的安装}
    这道题期末不考.
\end{frame}
\end{document}