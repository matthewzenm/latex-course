\documentclass{article}

\usepackage[paper=b5paper]{geometry}
\usepackage{amsmath}
\usepackage{amssymb}
\usepackage{amsthm}

\theoremstyle{plain}
\newtheorem{theorem}{Theorem}

\title{Scalar Curvature}
\author{Your name}
\date{\today}

\begin{document}
\maketitle
\section{Introduction}
It is known that scalar curvature controls the volume of geodesic ball.

\section{Formula}
\begin{theorem}
    The asymptotic expression of volume of geodesic ball is:
    \[\operatorname{Vol}(B_r(p))=\omega_nr^n\left(1-\frac{\operatorname{Scal}(p)}{6(n+2)}+O(r^3)\right),(r\to 0).\]
\end{theorem}
\begin{proof}
    See Gallot, et.\ al., \emph{Riemannian Geometry}.
\end{proof}

\end{document}